\documentclass[a4paper,12pt]{article}
\usepackage[utf8x]{inputenc}
\usepackage[T2A]{fontenc}
\usepackage[russian]{babel}
\usepackage[pdf]{graphviz}
\usepackage{float}
\usepackage{amsmath}
\usepackage{listings}
\usepackage{xcolor}
\usepackage[pdftex,unicode]{hyperref}
\usepackage{amsfonts}
\usepackage{algorithm}
\usepackage{algpseudocode}
\usepackage{enumitem}
\newcommand\tab[1][1cm]{\hspace*{#1}}

\lstset{
    frame=tb,
    tabsize=4,
    showstringspaces=false,
    numbers=left,
    commentstyle=\color{green},
    keywordstyle=\color{blue},
    stringstyle=\color{red}
}

\newtheorem{algo}{Алгоритм}
\newtheorem{definition}{Опредление}
\newtheorem{comment}{Комментарий}
\newtheorem{problem}{Задача}
\newtheorem{thesis}{Тезис}

\begin{document}
Введем некоторые определения:
\begin{itemize}
\item $at(\phi)$ - множество атомов в формуле $\phi$
\item $at_i(\phi)$ - некоторый заданный порядок этих атомов
\item Атом $a$ сопоставим с булевой переменной $e(a)$ (такую процедуру будем называть булевское кодирование)
\item $e(t)$ - булева формула, полученную булевым кодированием каждого атома формулы $t$ (такую формулу будем называть
пропозиционным скилетом формулы $t$)
\end{itemize}
Рассмотрим следующий пример: $\phi := x = y \vee x = z$. Тогда $at(\phi) = \{at_1, at_2\}$, $at_1 := x = y$, $at_2 := x = z$.
Сопоставим $x = y$ булеву переменную $e(x = y) = b_1$ и $x = z$ булеву переменную $e(x = z) = b_2$. Получаем
$e(\phi) = b_1 \vee b_2$.\newline
Пусть $\alpha$ - некоторая (возможно, частичная) оценка формулы $e(\phi)$, например для формулы выше пусть $\alpha = \{e(at_1)
\rightarrow FALSE, e(at_2) \rightarrow TRUE\}$.\newline
\begin{equation*}
Th(at_i, \alpha) =
\begin{cases}
at_i &\text{если $\alpha(at_i) = TRUE$}\\
\lnot at_i &\text{иначе}
\end{cases}
\end{equation*}
$Th(\alpha) = \{Th(at_i, \alpha)| для всех e(at_i), оцененных \alpha\}$, $\overline{Th(\alpha)}$ - конъюнкция всех элементов их
$Th(\alpha)$\newline
Пусть нам данн алгоритм (назовём его $Deduction$), который может решить $\overline{Th(\alpha)}$. Опишем алгоритм решение
SMT-формул, имея $Deduction$:\newline
\begin{itemize}
\item Вычислим $B = e(\phi)$
\item Отправим $B$ SAT-решателю
\item Если получили "UNSAT", то возвращаем "UNSAT"
\item Если получили "UNKNOWN", то возвращаем "UNKNOWN"
\item Если получили некоторую $\alpha$, то вычисляем $Deduction$ от $\overline{Th(\alpha)}$
\item Если получили "SAT", то возвращаем "SAT"
\item Если получили "UNKNOWN", то возвращаем "UNKNOWN"
\item Если получили "UNSAT", то $B = B \wedge$ "блокирующий дизюнкт" $t$ и начинаем со второго пункта
\end{itemize}
Опишем подробнее свойства "блокирующего дизюнкта" (так же называют леммой):\newline
\begin{itemize}
\item $t$ - тавтология в $T$
\item $t$ - состоит только из атомов из $\phi$
\item $t$ - "блокирует" $\alpha$, т.е. при отправлении $B$ SAT-решателю мы не сможем снова получить $\alpha$
\end{itemize}
Предположим, что в предыдущем примере оценка оказалась "UNSAT", тогда можно привести такой "блокирующий дизюнкт": $t := e(at_1)
\vee \lnot e(at_2)$ (т.е. мы просто взяли дизъюнкцию отрицаний атомов). Есть и другие способы найти блокирующий дизюнкт,
например, часто можно взять дизъюнкцию части отрицаний атомов, если они "UNSAT" в данной теории.\newline

Есть более эффективные алгоритмы решения SMT задач, но для этого нам нужно изучить, как работает SAT решатель. В процессе работы
ленивого алгоритма может возникнуть много блокирующих дизюнктов. Многие теории можно свести к SAT задаче, но обычно существуют
более эффективные алгоритмы решения теорий.\newline

Решим с помощью SMT-решателя задачу с прошлой лекции: задача о расскарске графа. Имеется граф $G = (V, E)$. Можно ли его вершини
расскрасить в $k$ цветов так, чтобы никакие соседние вершины не были раскрашены в один и тот же цвет? Постороим следующую
систему уравнений:\newline
\begin{itemize}
\item $x_i$ - целые
\item $1 \le x_i \le c$
\item $x_i \neq x_j$ для $(x_i, x_j) \in E$
\end{itemize}
Для решения такой системы нам понадобится SMT-решатель, умеющий разрешать теорию равенства.\newline
В качестве упражнения предоставляются следующие задачи:
\begin{itemize}
\item Судоку (заполнить поле числами)
\item Сапер (указать, на какую клетку можно "кликнуть")
\end{itemize}

В конце лекции разбрем следующую задачу - даны две программы, a и b:\newline
Программа a:\newline
int i, out\_a = in;\newline
for (int i = 0; i < 2; ++i) out\_a = out\_a * i;\newline
return out\_a;\newline
Программа b:\newline
int out\_b = (in * in) * in;\newline
return out\_b;\newline
Нужно проверить, являются ли они эквивалентными друг другу. Для этого построим формулы, описывающие связь между входом и выходом
программ:\newline
$\phi_a := (out\_a_0 = in\_a_0) \wedge (out\_a_1 = out\_a_0 * in\_a_0) \wedge (out\_a_2 = out\_a_1 * in\_a_0)$\newline
$\phi_b := out\_b_0 = (in\_b_0 * in\_b_0) * in\_b_0$\newline
Чтобы программы были эквивалентны, должно выполняться:\newline
$(in\_a_0 = in\_b_0) \wedge \phi_a \wedge \phi_b \rightarrow out\_a_2 = out\_b_0$\newline
Проверку такой формулы на выполнимость может осуществить SMT-решатель, умеющий разрешать теорию равенства.\newline
\end{document}
